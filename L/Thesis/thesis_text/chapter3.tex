%%%%%%%%%%%%%%%%%%%%%%%%%%%%%%%%%%%%%%%%%%%%%%%%%%%%%%%%%%%%%%%%%%% 
%                                                                 %
%                            CHAPTER                              %
%                                                                 %
%%%%%%%%%%%%%%%%%%%%%%%%%%%%%%%%%%%%%%%%%%%%%%%%%%%%%%%%%%%%%%%%%%% 
\chapter{Research and Investigation}
\section{Introduction}
This chapter serves as an explanation on certain choices that are made.
A chapter like this is in my opinion justified because of a few reasons.
Doing research is not just answering a question, there is a complete process in which a lot of choices have to be made,
and in certain moments, choices need to be revisit. 
Often more than one wrong choice is made but making that wrong choice still contributes to the final result.
In the next sections I will describe my process throughout the past year.
\section{Timeline}
\subsection{Initial Research}
My starting knowledge in the field of machine learning was close to zero.
Therefore the initial research phases consisted mainly of studying the basics of machine learning and different kinds of implementations.
With the basic knowledge built up, I started with the LS-SVM algorithm itself and playing around with the toolbox.
\subsection{Parallel Execution Research}
I started with the idea of running Matlab server on a remote machine and try to make it work.
In my mind this was very useful because the end goal for me at that point was executing parallel code on a remote machine.
However in this process, I had a lot of difficulties with getting the Matlab server to run and to execute code.
\par 
Because it was harder than expected, I started exploring other options.
I knew that a lot of heavy computations were matrix computations.
Matlab is very optimized in matrix computations but not so much in memory management.
The algorithm requires a lot of passing around matrices from one function to another.
If there is one computer language that is perfect for controlling memory and making sure nothing is copied when you don't want to, it is C++.
As a benefit frameworks like OpenMP exist for CPU parallel execution and intelMKL and others for optimized matrix operations on the CPU.
By combining these facts I made the decision to translate the needed functions from Matlab to C++, and to start the development of a C++ FS LS-SVM library.
\subsection{Development}
In the first phase I started with using the Matlab coder toolbox to translate the existing Matlab scripts in the LS-SVM toolbox to c-code.
I quickly found out that:
\begin{enumerate}
	\item Most scripts needed a lot of alterations because the optimized fixed size least square support vector algorithm was not or barely not implemented in the toolbox.
	\item In order to make use of the OpenMp toolbox everything had to be rewritten.
\end{enumerate} 
Manual translation seemed a better option.
I completely separated all the linear algebra from the algorithm in two different libraries.
With the logic that my personal linear algebraic library would be a bridge between an existing optimised framework like intelMKL and the LS-SVM library.
I followed an introduction course to OpenMP at the ICTS center of the KULeuven to get a good start.
OpenMP is, in my opinion, a very good framework that easily gives you a lot of control over parallel threads and making scalability very easy without a lot of environment specifics.
The LS SVM C++ library was quiet easy to develop but the integration with linear algebraic libraries was much more difficult than expected.
Big problems arose with compatibility of the compiler, the use of OpenMp together with the other frameworks and more.
\par 
The compiler settings, paths and cmake files were becoming so environment specific that porting the software from my local machine to a remote machine were becoming impossible.
That together with the fact that without a working optimized linear algebraic framework, it would not be competitive to Matlab execution.
Even with better memory management.
\par 
Those problems forced me to drop all the C++ code and return to Matlab.
The remaining problem: remote execution of the code.
Fortunately, Matlab offers remote execution together with commercial partners in a way that is independent from any environment variables. 
I started adjusting all the needed scripts to optimized FS LS-SVM and creating working models.
\section{Personal Note}
I am aware that this is an odd chapter in a thesis. 
However, for me it is an important one. 
I worked for almost nine months on the C++ library before throwing it all away, meaning that my actual development process was mostly done there and not in the used Matlab code.
This is ofcourse not a waste, I learned C++ programming in a professional way and I am comfortable to say that my knowledge of the OpenMP framework, as well as linear algebraic frameworks as intelMKL, is more than decent.
The biggest downside is that I have investigated less than I hoped to. 
As you will see, there are some very interesting future extensions described in chapter 7 that I actually was hoping to investigate myself.
If there is any interest in my C++ code, I am always willing to open the repositories.


